\section*{Recursividad vs. Iteración}

El objetivo de esta parte de Trabajo Práctico es revisar los temas
abordados durante este período sobre \emph{pruebas del sotfware
basado en herramientas estándar} para comparar el comportamiento
de algoritmos \emph{recursivos} frente a su versión \emph{iterativa}.
Además, se realizará la comparación de las versiones \emph{recursivas}
para los casos de \emph{Evaluación Ansiosa} (\emph{eager}) y
\emph{Evaluación Perezosa} (\emph{lazy}). Finalmente, se hará un estudio de los resultados obtenidos.


\subsection*{Caso 1: Recursión Simple}

En este caso vamos a utilizar el cálculo del \emph{Factorial} de un número para comparar la eficiencia entre la versión recursiva y la versión iterativa.

La definición matemática del \emph{Factorial} de un número nos dice que:
\begin{equation*}
    n! = \left\{ \begin{array}{ll}
        1                & \text{para } n = 0 \text{ o } n = 1 \\
        n \cdot (n - 1)! & \text{para } n > 0
    \end{array} \right.
\end{equation*}

La función recursiva que resuelve esto es una codificación trivial de la función matemática y podría escribirse de esta forma:

\lstinputlisting[style=editor,linerange={26-33}]{../src/main/cpp/Recursividad/Factorial.cpp}

La versión iterativa podría escribirse de esta forma:

\lstinputlisting[style=editor,linerange={25-33}]{../src/main/cpp/Recursividad/Factorial-it.cpp}

Testee las dos versiones para diferentes valores factoriales, trace las curvas con los
tiempos medidos y responda las siguientes preguntas:
\begin{enumerate}
    \item ¿Cuál de las dos es más eficiente?
    \item ¿Cuál de las dos es más programable?
    \item ¿El comportamiento de ambas versiones es lineal o exponencial?
\end{enumerate}


\subsection*{Caso 2: Recursión Doble}

En este caso vamos a utilizar el cálculo de los \emph{Números de Fibonacci}
para comparar la eficiencia entre la versión recursiva y la versión iterativa.

La definición matemática de los \emph{Números de Fibonacci} nos dice que:
\begin{equation*}
    F_n = \left\{ \begin{array}{ll}
        1                & \text{para } n = 1 \text{ o } n = 2 \\
        F_{n-1} + F_{n-2} & \text{para } n > 2
    \end{array} \right.
\end{equation*}

La función recursiva que resuelve esto es una codificación trivial de la función matemática y podría escribirse de esta forma:

\lstinputlisting[style=editor,linerange={25-33}]{../src/main/cpp/Recursividad/Fibonacci.cpp}

La versión iterativa podría escribirse de esta forma:

\lstinputlisting[style=editor,linerange={25-36}]{../src/main/cpp/Recursividad/Fibonacci-it.cpp}

Testee las dos versiones para diferentes valores factoriales, trace las curvas con los
tiempos medidos y responda las siguientes preguntas:
\begin{enumerate}
    \item ¿Cuál de las dos es más eficiente?
    \item ¿Cuál de las dos es más programable?
    \item ¿El comportamiento de ambas versiones es lineal o exponencial?
\end{enumerate}


\subsection*{Caso 3: Combinaciones}

En este caso se debe explorar la implementación del cálculo de las
\emph{Combinaciones sin repetición}, cuya definición matemática es:
\begin{equation*}
    C(n, r) = \begin{pmatrix} n \\ r \end{pmatrix} = \frac{n!}{r!(n-r)!} \quad \text{para } n \geq r
\end{equation*}
donde
\begin{equation*}
    \begin{pmatrix}
        n \\ n
    \end{pmatrix} =
    \begin{pmatrix}
        n \\ 0
    \end{pmatrix} = 1    
\end{equation*}

Se puede modificar la expresión matemática para que quede expuesta una versión
recursiva:
\begin{equation*}
    \begin{pmatrix} n \\ r \end{pmatrix} =
    \begin{pmatrix}
        n - 1 \\ r
    \end{pmatrix} +
    \begin{pmatrix}
        n - 1 \\ r - 1
    \end{pmatrix}
\end{equation*}

En este caso se están presentando la versión iterativa de la solución como
así también un método matemático equivalente para solucionar el problema.
Se pide analizar los siguientes casos:
\begin{enumerate}
    \item Combinaciones con Factorial Recursivo,
    \item Combinaciones con Factorial Iterativo,
    \item Combinaciones versión Recursiva.
\end{enumerate}


\subsection*{Caso 4: Evaluación Ansiosa vs. Perezosa}

La mayoría de los lenguajes y compiladores que utilizamos está pensado para
el paso de variables por valor.  Este tipo de ejecución se denomina
\emph{Evaluación Ansiosa} (\emph{Eager Evaluation}).  La evaluación ansiosa
tiene como inconveniente que no se puede utilizar convenientemente la
recursividad por limitación del tamaño de pila realizable y el alto consumo
de recursos que requiere.

Por el contrario, los lenguajes y compiladores que instrumentan la
\emph{Evaluación Perezosa} (\emph{Lazy Evaluation}) basan su funcionamiento
en la dilación por el mayor tiempo posible de la evaluación de los cálculos
necesarios.  Este mecanismo permite implementar algoritmos recursivos y sobre
conjuntos de datos \emph{virtualmente infinitos} sin inconveniente y a un costo computacional razonable.

En este caso se deberá comparar el comportamiento de los algoritmos
\emph{Factorial de un número} y \emph{Secuencia de Fibonacci} implementados
con evaluación perezosa frente a los tiempos medidos con los algoritmos de
los casos anteriores.

El \emph{Factorial de un número} en Haskell se puede definir como
\lstinputlisting[style=editor,linerange={3-7}]{../src/main/haskell/Factorial.hs}

La \emph{Secuencia de Fibonacci} en Haskell se puede definir como
\lstinputlisting[style=editor,linerange={3-8}]{../src/main/haskell/Fibonacci.hs}

No olvide implementar los algoritmos para \emph{Combinaciones sin repetición}
en Haskell (solo versiones recursivas).

\subsection*{Nota}

Para la realización de este estudio se aconseja desarrollar \emph{shell scripts}
que le permitan la repetibilidad de las pruebas a realizar.

Se pide que se midan los tiempos de ejecución de cada versión para
diferentes valores de entrada. También se pide que se estudie el número
de llamadas a las funciones para cada versión. Estos resultados deberán
verse reflejados en gráficos comparativos y árboles de llamadas.

No olvide hacer las comparaciones para la compilación sin y con optimizaciones
del compilador (\verb|gcc| y \verb|ghc| con \verb|-O1|, \verb|-O2| y
\verb|-O3|) 

A la hora de entregar los resultados envíe el informe en formato PDF y
los código fuente utilizados tanto de los programas como de los \emph{shell scripts}.

