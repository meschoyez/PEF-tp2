\section*{Reutilización de Objetos}

\subsection*{Garbage Collector}

Los lenguajes que implementan el \emph{garbage collector} fueron
diseñados para disminuir la carga al programador al desligarlo de
controlar la asignación/liberación de memoria. Esto hace que el
programador solo deba preocuparse de pedir la creación de los objetos,
utilizarlos y luego simplemente desreferenciarlos. La tarea de limpieza
de la memoria queda a cargo de otro. Sin embargo, la ``recolección de
(memoria) basura'' tiene costo computacional.

En los programas cuya memoria es bastante estática a lo largo de su
ejecución, la tareas del \emph{garbage collector} es baja. En cambio,
si la tasa de asignación/liberación de memoria es muy alta, el
``recolector de basura'' tendrá mucho mayor trabajo, produciendo picos
de consumo de recursos a la hora de realizar esta tarea.

La activación de la liberación de memoria tiene diferentes políticas y
el programador puede pedir que se utilice alguna en particular. Sin
embargo, no está a cargo del programador la liberación de memoria, por
lo cual, se hará cuando el \emph{garbage collector} determine que es
momento apropiado.


\subsection*{Reutilización de Objetos}

Si estamos frente a un programa que requiere gran generación/destrucción
de objetos se puede optar por una política de reutilización de los objetos.
En este caso, en lugar de crear un nuevo objeto cada vez que se lo necesite
se reinicializará un objeto existente en desuso para adaptarlo a la nueva
tarea. De esta forma se logran dos objetivos:
\begin{itemize}
    \item se evita el costo computacional requerido por el constructor del
    objeto y
    \item se libera al \emph{garbage collector} de su tarea de llamada al
    destructor y liberación de la memoria.
\end{itemize}
Por supuesto, para que la reutilización de objetos sea factible deberá mantenerse una referencia a los objetos en desuso dentro de algún tipo
de colección.


\subsection*{Conigna}

Para la realización de este estudio se aconseja trabajar sobre la IDE
NetBeans utilizando el \emph{profiler} provisto.

Se pide que estudie el comportamiento del programa “Zorros y Conejos”.
Este programa es un proyecto BlueJ que se entrega convertido en un
proyecto con Maven para poder trabajar. El programa es una simulación
que requiere una alta tasa de creación/destrucción de objetos (zorros
y conejos) para estudiar el comportamiento de sus nacimientos/muertes.

Una vez estudiado el programa deberá planificar e implementar una
propuesta de reutilización de objetos.

Se pide que se midan los tiempos de ejecución de cada versión. Estos
resultados deberán verse reflejados en gráficos comparativos y árboles
de llamadas.

